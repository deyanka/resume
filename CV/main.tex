\documentclass[10pt, letterpaper]{article}

% Packages:
\usepackage[
    ignoreheadfoot, % set margins without considering header and footer
    top=2 cm, % seperation between body and page edge from the top
    bottom=2 cm, % seperation between body and page edge from the bottom
    left=2 cm, % seperation between body and page edge from the left
    right=2 cm, % seperation between body and page edge from the right
    footskip=1.0 cm, % seperation between body and footer
    % showframe % for debugging 
]{geometry} % for adjusting page geometry
\usepackage{titlesec} % for customizing section titles
\usepackage{tabularx} % for making tables with fixed width columns
\usepackage{array} % tabularx requires this
\usepackage[dvipsnames]{xcolor} % for coloring text
\definecolor{primaryColor}{RGB}{0, 0, 0} % define primary color
\usepackage{enumitem} % for customizing lists
\usepackage{fontawesome5} % for using icons
\usepackage{amsmath} % for math
\usepackage[
    pdftitle={Kravchenko Diana's CV},
    pdfauthor={Kravchenko Diana},
    pdfcreator={LaTeX with RenderCV},
    colorlinks=true,
    urlcolor=primaryColor
]{hyperref} % for links, metadata and bookmarks

\usepackage[utf8]{inputenc}     % for UTF8 codepage in Windows
\usepackage[russian]{babel}

\usepackage[pscoord]{eso-pic} % for floating text on the page
\usepackage{calc} % for calculating lengths
\usepackage{bookmark} % for bookmarks
\usepackage{lastpage} % for getting the total number of pages
\usepackage{changepage} % for one column entries (adjustwidth environment)
\usepackage{paracol} % for two and three column entries
\usepackage{ifthen} % for conditional statements
\usepackage{needspace} % for avoiding page brake right after the section title
\usepackage{iftex} % check if engine is pdflatex, xetex or luatex

% Ensure that generate pdf is machine readable/ATS parsable:
\ifPDFTeX
    \input{glyphtounicode}
    \pdfgentounicode=1
    \usepackage[T1]{fontenc}
    \usepackage[utf8]{inputenc}
    \usepackage{lmodern}
\fi

\usepackage{charter}

% Some settings:
\raggedright
\AtBeginEnvironment{adjustwidth}{\partopsep0pt} % remove space before adjustwidth environment
\pagestyle{empty} % no header or footer
\setcounter{secnumdepth}{0} % no section numbering
\setlength{\parindent}{0pt} % no indentation
\setlength{\topskip}{0pt} % no top skip
\setlength{\columnsep}{0.15cm} % set column seperation
\pagenumbering{gobble} % no page numbering

\titleformat{\section}{\needspace{4\baselineskip}\bfseries\large}{}{0pt}{}[\vspace{1pt}\titlerule]

\titlespacing{\section}{
    % left space:
    -1pt
}{
    % top space:
    0.3 cm
}{
    % bottom space:
    0.2 cm
} % section title spacing

\renewcommand\labelitemi{$\vcenter{\hbox{\small$\bullet$}}$} % custom bullet points
\newenvironment{highlights}{
    \begin{itemize}[
        topsep=0.10 cm,
        parsep=0.10 cm,
        partopsep=0pt,
        itemsep=0pt,
        leftmargin=0 cm + 10pt
    ]
}{
    \end{itemize}
} % new environment for highlights


\newenvironment{highlightsforbulletentries}{
    \begin{itemize}[
        topsep=0.10 cm,
        parsep=0.10 cm,
        partopsep=0pt,
        itemsep=0pt,
        leftmargin=10pt
    ]
}{
    \end{itemize}
} % new environment for highlights for bullet entries

\newenvironment{onecolentry}{
    \begin{adjustwidth}{
        0 cm + 0.00001 cm
    }{
        0 cm + 0.00001 cm
    }
}{
    \end{adjustwidth}
} % new environment for one column entries

\newenvironment{twocolentry}[2][]{
    \onecolentry
    \def\secondColumn{#2}
    \setcolumnwidth{\fill, 4.5 cm}
    \begin{paracol}{2}
}{
    \switchcolumn \raggedleft \secondColumn
    \end{paracol}
    \endonecolentry
} % new environment for two column entries

\newenvironment{threecolentry}[3][]{
    \onecolentry
    \def\thirdColumn{#3}
    \setcolumnwidth{, \fill, 4.5 cm}
    \begin{paracol}{3}
    {\raggedright #2} \switchcolumn
}{
    \switchcolumn \raggedleft \thirdColumn
    \end{paracol}
    \endonecolentry
} % new environment for three column entries

\newenvironment{header}{
    \setlength{\topsep}{0pt}\par\kern\topsep\centering\linespread{1.5}
}{
    \par\kern\topsep
} % new environment for the header

\newcommand{\placelastupdatedtext}{% \placetextbox{<horizontal pos>}{<vertical pos>}{<stuff>}
  \AddToShipoutPictureFG*{% Add <stuff> to current page foreground
    \put(
        \LenToUnit{\paperwidth-2 cm-0 cm+0.05cm},
        \LenToUnit{\paperheight-1.0 cm}
    ){\vtop{{\null}\makebox[0pt][c]{
        \small\color{gray}\textit{Last updated in September 2024}\hspace{\widthof{Last updated in September 2024}}
    }}}%
  }%
}%

% save the original href command in a new command:
\let\hrefWithoutArrow\href

% new command for external links:


\begin{document}
    \newcommand{\AND}{\unskip
        \cleaders\copy\ANDbox\hskip\wd\ANDbox
        \ignorespaces
    }
    \newsavebox\ANDbox
    \sbox\ANDbox{$|$}

    \begin{header}
        \fontsize{25 pt}{25 pt}\selectfont Кравченко Диана

        \vspace{5 pt}

        \normalsize
        \mbox{Москва}%
        \kern 5.0 pt%
        \AND%
        \kern 5.0 pt%
        \mbox{\hrefWithoutArrow{mailto:ddkravchenko@edu.hse.ru}{ddkravchenko@edu.hse.ru}}%
        \kern 5.0 pt%
        \AND%
        \kern 5.0 pt%
        \mbox{\hrefWithoutArrow{tel:+7-925-369-50-79}{+7 (925) 369 50 79}}%
        \kern 5.0 pt%
        \AND%
        \kern 5.0 pt%
        \mbox{\hrefWithoutArrow{https://github.com/deyanka}{github.com/deyanka}}%
    \end{header}

    \vspace{5 pt - 0.3 cm}





        

    \section{Образование}

   
        \begin{twocolentry}{
            Сентябрь 2022 – Май 2026
        }
            \textbf{НИУ ВШЭ, Москва}, 
            \newline
            бакалаврская программа 'Вычислительные социальные науки' с углубленным изучением прикладной математики и информатики (ПМИ) \end{twocolentry}

        \vspace{0.10 cm}
        \begin{onecolentry}
            \begin{highlights}
                \item GPA: 9/10 (\href{https://www.hse.ru/ba/compsocsci/ratings?course=3&from=1014855644}{рейтинг})
                \item \textbf{Релевантные курсы:} 
                \newline
                Теория вероятности, Статистика, Эконометрика, Анализ временных рядов;
                \newline
                Python, C++;
                \newline
                Алгоритмы и структуры данных, Машинное обучение, Глубинное обучение.
            \end{highlights}
        \end{onecolentry}

    \section{Навыки}
    

        \begin{tabular}{ @{} >{\bfseries}l @{\hspace{6ex}} l }
            Языки программирования & 
            \\
            Продвинутый уровень: &  Python, SQL, STATA;\\
            Базовый уровень: &  R, C++, C, RISC-V;\\
            \\
            Базы данных: & SQL, Excel;\\
            Анализ данных: & Pandas, NumPy, StatsModels, SciPy;\\
            Построение моделей: & Scikit-Learn, PyTorch;
            \\ 
            \\
            Языки &
            \\
            Английский & C1;\\
            Испанский & B2;\\
            
            \end{tabular}\\

        \vspace{0.2 cm}



    
    \section{Проекты}



        
        \begin{twocolentry}{
            Июнь 2024
        }
            \textbf{Разработка методики для анализа аффективной поляризованности на основе публикаций во Вконтакте} -- Москва, НИУ ВШЭ\end{twocolentry}

        \vspace{0.10 cm}
        \begin{onecolentry}
            \begin{highlights}
                \item Был пересмотрен подход к измерению аффективной поляризованности онлайн и были выведены собственные метрики, позволяющие учитывать большее число факторов.
                \item Была обучена нейронная сеть на основе RuBert-tiny для разметки текстовых данных. 
                \item Предложенный подход позволил провести анализ поляризованности на большом массиве данных (более миллиона публикаций).
                \item Работа получила лауреатство в конкурсе НИРС по направлению политология. (\href{https://www.hse.ru/nirs/nirs2024_results?ysclid=m82zlkb56l414299801}{результаты})
            \end{highlights}
        \end{onecolentry}



        \vspace{0.2 cm}

        \begin{twocolentry}{
            Декабрь 2024
        }
            \textbf{Debate around feminism in Telegram. Analysing structural polarisation online} -- Венеция, Ca'Foscari University of Venice\end{twocolentry}

        \vspace{0.10 cm}
        \begin{onecolentry}
            \begin{highlights}
                \item Был проведен сетевой анализ отобранных Telegram-каналов, в результате чего были оценены структуры информационных потоков по исследуемой теме.
                \item С помощью модели RuBert было произведено тематическое моделирование, что позволило oценить различия выявленных на предыдущем шаге сообществ. 
                \item Был проведен сентимент анализ для оценки аффектов пользователей.
            \end{highlights}
        \end{onecolentry}
        \vspace{0.2 cm}
                \begin{twocolentry}{
            Февраль 2025
        }
            \textbf{Анализ структурной поляризованности в телеграме по вопросу выборов в США 2024} -- Москва, НИУ ВШЭ\end{twocolentry}

        \vspace{0.10 cm}
        \begin{onecolentry}
            \begin{highlights}
                \item Был использован label propogation алгоритм для оценки связей между каналами
                \item Используется Monte-Carlo SSA (MCSSA) для оценки изменений размеров выведенных кластеров относительно значимых для рассматриваемого феномена событий. 

            \end{highlights}
        \end{onecolentry}




    

\end{document}